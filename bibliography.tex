\begin{thebibliography}

\bibitem{matthews2008intelturbomem}
Matthews J, Trika S, Hensgen D, Coulson R, Grimsrud K. Intel Turbo Memory: Nonvolatile disk caches in the storage hierarchy of mainstream computer systems. ACM Trans on Storage, 2008, Article 4, 24 pages.

\bibitem{jimgray2003cloud}
Jim Gray. What next?: A dozen information-technology research goals. Journal of the ACM, 2003, 50: 41–57. Turing Award lecture.

\bibitem{morris2003evostorage}
R.J.T.Morris, B.J.Truskowski. The evolution of storage. IBM SYSTEMS JOURNAL, 2003, Vol.42, No.2.

\bibitem{libo2010cacheforssd}
Li Bo, Xie Changsheng, Wang Fen, Zhao Xiaogang. New Cache Replacement Algorithm for Solid-state Drive. Computer Science, 2010, Vol.37, No.8.

\bibitem{henry2014ssdprice}
Henry Newman. SSD vs HDD Pricing: Seven Myths That Need Correcting. Enterprise Storage Forum, 2014.

\bibitem{taeho2006flashcache}
Taeho Kgil, Trevor Mudge. FlashCache: A NAND Flash Memory File Cache for Low Power Web Servers. Advanced Computer Architecture Laboratory, CASES’06, October 23–25, 2006.

\bibitem{vfcache}
中国电子科技集团公司第五十四研究所. LSI PCIe闪存适配器助力Cisco和EMC实现应用加速. 计算机与网络, 2012, 38(11).

\bibitem{smartcache}
李霁蓉. SmartCache技术在校园视频监控系统中的应用. 消费电子, 2013, 18:13-14.

\bibitem{lsiraidcache}
张明. LSI推出MegaRAID CacheCade Pro 2.0 SSD高速缓存软件. 电子工业出版社, 电子与电脑, 2011, 9:90.

\bibitem{enhanceio}
Michael Larabel. EnhanceIO: New Solid State Drive Caching For Linux. Phoronix Premium, 13 January 2013.

\bibitem{fusionio}
王珩. Fusion-io实现每秒12GB视觉特效数据传输. 计算机与网络, 2013, 16:58-58.

\bibitem{hdd2009}
Y. Shiroishi, K. Fukuda, I. Tagawa, H. Iwasaki, S. Takenoiri, H. Tanaka, H. Mutoh, N. Yoshikawa. Future Options for HDD Storage. IEEE Transactions on Magnetics, 2009, Vol 45, Issue 10.

\bibitem{ssd2009}
王伟能, 王鹤群. 固态硬盘概述. 记录媒体技术, 2009, Vol 1.

\bibitem{cache2011}
汪小林, 赖荣凤, 王振林, 罗英伟, 李晓明. 基于SSD高速缓存的桌面虚拟机交互性能优化方法. 计算机应用与软件, 2011, Vol 28, No 11.

\bibitem{zhuqing2013hybrid}
Zhu Qing, Li Xiaoyong. A Review on Hybrid Storage. Microcomputer Application, 2013, Vol.29, No.2:33-38.

\bibitem{sdramcache2002}
苏海冰, 吴钦章. 用SDRAM在高速数据采集和存储系统中实现海量缓存. 光学和精密工程, 2002, Vol 10, No 5.

\bibitem{nvramcache2013}
Sheng Qiu, A. L. Narasimha Reddy. NVMFS: A Hybrid File System for Improving Random Write in NAND-flash SSD. IEEE 29th Symposium on Mass Storage Systems and Technologies (MSST). 2013.

\bibitem{linlin2011}
Lin Lin, Yifeng Zhu, JianhuiYue, Zhao Cai, Bruce Segee. Hot Random Off-loading: A Hybrid Storage System With Dynamic Data Migration. IEEE 19th International Symposium on Modeling, 2011, Analysis Simulation of Computer and Telecommunication Systems (MAS-COTS).

\bibitem{wudong2013raid1}
Wu Dong. A Research of RAID1 Technology Based on SSD Cache. Thesis for the Degree of Master of Engineering. 2013. 1-54.

\bibitem{zhuqing2012hybrid}
Zhu Qing. Study on Hybrid Storage Systems. Thesis for the Degree of Master of Engineering. 2012. 1-56.

\bibitem{nvramcache1992}
Mary Baker, Satoshi Asami, Etienne Deprit, John Ousterhout, and Margo Seltzer. Non-volatile memory for fast, reliable file systems. International Conference on Architectural Support for Programming Languages and Operating Systems (ASPLOS), October 1992, pages 10–22, Boston, MA.

\bibitem{wdm2001}
张伟, 张云麟. Windows驱动程序模型的设计与开发. 重庆邮电学院学报:自然科学版, 2001, 第3期 88-91.

\bibitem{filterdrv2004}
梁德祥. 利用过滤层驱动程序实现移动硬盘加密. 盐城工学院学报(自然科学版), 2004, Vol 17, No 3.

\bibitem{LRU}
Marek Chrobak, John Noga. LRU is Better than FIFO. Proceedings of the ninth annual ACM-SIAM symposium on Discrete algorithms, 1998.

\bibitem{LFU}
黄秀荪, 仇玉林, 叶青. LFU算法的ASIC实现. 电子器件, 2007, 第1期, 237-240.

\bibitem{cachemap2013}
沈秀红. 基于基地址寄存器映射的数据高速缓存设计研究. 浙江大学集成电路工程硕士学位论文. 2013.

\bibitem{bplustree2012}
长孙妮妮, 张毅坤, 华灯鑫, 邹子夏, 陈浩. 一种基于B+树的混合索引结构. 计算机工程, 2012, Vol 38, No 14.

\bibitem{writeback2014}
吴纪锋, 吴文江, 秦承刚. 基于页面优先级策略的文件回写机制研究. 小型微型计算机系统, 2014, Vol 35, No 1.

\bibitem{writethrough2010}
梅魁志, 李国辉, 张斌. 一种面向写穿透Cache的写合并设计及验证. 西安交通大学学报, 2010, Vol 44, No 4, Apr.

\bibitem{writeback2008}
林伟, 叶笑春, 宋风龙, 张浩. 众核处理器中使用写掩码实现混合写回/写穿透策略. 计算机学报, 2008, Vol 31, No 11, Nov.

\bibitem{ssddesign2008}
Nitin Agrawal, Vijayan Prabhakaran, Ted Wobber, John D. Davis, Mark Manasse, and Rina Panigrahy. Design tradeoffs for SSD performance. USENIX Annual Technical Conference, 2008, pages 57–70, Boston, MA.

\bibitem{integssdhdd2011}
Yang Qing, Jin Ren. I-CASH: Intelligently Coupled Array of SSDs and HDDs. in: Proceeding of HPCA'11. Texas: IEEE Computer Society Press, 278-289, 2011.

\bibitem{cacheforflash2012}
Oh Y, Choi J, Lee D. Caching Less for Better Performance: Balancing Cache Size and Update Cost of Flash Memory Cache in Hybrid Storage Systems. FAST'12, San Jose: ACM Press, 313-326, 2012.

\bibitem{hystor2011}
Feng Chen, David Koufaty, Xiaodong Zhang. Hystor: Making the Best Use of Solid State Drives in High Performance Storage Systems. ICS (International Conference on Supercomputing), 22–32, 2011.

\bibitem{icash2011}
Jin Ren, Qing Yang. I-CASH: Intelligently Coupled Array of SSD and HDD. IEEE 17th International Symposium on High Performance Computer Architecture (HPCA), 2011.

\bibitem{computerarch2011}
李学干. 计算机系统结构. 西安电子科技大学出版社, ISBN: 9787560601397, 2011.

\bibitem{ARC}
Nimrod Megiddo, Dharmendra S. Modha. ARC: A Self-Tuning, Low Overhead Replacement Cache. FAST ’03: 2nd USENIX Conference on File and Storage Technologies, 2003.

\bibitem{windowsdrv2008}
张帆. Windows驱动开发技术详解. 电子工业出版社, ISBN: 9787121068461, 2008.

\bibitem{winkernprg2008}
谭文, 邵坚磊. 天书夜读-从汇编语言到Windows内核编程. 电子工业出版社, ISBN: 9787121073397, 2008.

\bibitem{softwaredbg2008}
张银奎. 软件调试. 电子工业出版社, ISBN: 9787121064074, 2008.

\bibitem{softwaredbg2013}
张银奎. 格蠹汇编-软件调试案例集锦. 电子工业出版社, ISBN: 9787121196072, 2008.

\bibitem{findbugs2012}
Tobias Klein. 捉虫日记. 人民邮电出版社, ISBN:9787115290441, 2012.

\bibitem{dbghacks2011}
吉岡弘隆, 大和一洋, 大岩尚宏, 安部東洋, 吉田俊輔. Debug Hacks: 深入调试的技术和工具. 电子工业出版社, ISBN: 9787121140488, 2011.

\bibitem{deepinlinux2010}
Wolfgang Mauerer. 深入Linux内核架构. 人民邮电出版社, ISBN:9787115227430, 2010.

\end{thebibliography}
