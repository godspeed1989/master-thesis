%% ----------------------------------------------------------------------
%% START OF FILE
%% ----------------------------------------------------------------------
\chapter{总结与展望}
\label{cha:conclusions}

本论文首先介绍了固态硬盘和机械硬盘混合存储的技术背景和国内外存储设备厂商的相关研究工作,提出了利用固态硬盘空闲空间作为缓存,提升机械硬盘IO性能的解决方案。分析缓存技术的发展背景和支撑缓存系统的相关技术,分析不同类型存储介质的特点、混合存储技术的框架和缓存页面替换算法,进行论文方案可行的说明。

接下来,分析并详细介绍缓存系统实现中所用到的关键性技术,这些技术即包含学术和实践中已经成熟的技术:例如Windows系统的IO捕获、缓存映射策略;也包括论文提出的新技术解决方案:一种同时考虑缓存块最近最近访问时间和访问频率的缓存页面替换算法。

最后,论文给出了缓存系统的核心实现细节和性能测试结果。实现细节部分的介绍包含了缓存系统的整体设计架构、缓存算法通用接口和用户配置工具提供的参数。测试部分比较了本缓存系统和商业软件缓存系统对机械硬盘的性能提升效果。

\section{全文总结}
\label{sec:thesis_conclusion}
本论文的主要研究成果和贡献可归纳为以下几点。
\begin{enumerate}
\item
针对目前固态硬盘的价格相对机械硬盘仍然过高,存储中心大多同时使用机械硬盘和固态硬盘进行混合存储这一现状。经过综合分析了现存在的混合存储解决方案,本论文设计了一种基于固态硬盘的机械硬盘缓存存储存储架构,将固态硬盘的空闲空间用作机械硬盘上热数据的缓存,提高存储系统的IO性能。
\item
分析使用广泛的LRU、LFU等缓存页面替换算法,研究这些算法的冷热数据识别原理和实现逻辑。提出一种综合考虑最后一次访问时间和缓存块访问频数的缓存页面替换算法。
\item
在Windows平台下的实现本论文设想的混合存储缓存系统。针对大多数混合存储系统运行于Linux平台这一现状,在Windows平台使用WDM驱动开发框架实现混合存储系统,弥补Windows平台下可用的混合存储软件不足这一现状。
\item
在实现混合存储系统提供的多种缓存页面替换算法的过程中,本论文抽象出了一套可以用于所有缓存页面替换算法的通用函数接口。这套接口既降低了论文实现缓存系统时的代码复杂度,也可以作为设计页面替换算法时的功能性指导。
\item
经测试,论文实现的缓存系统的确可以带来机械硬盘性能提升,写穿模式的随机读性能有2-4倍的提升;写回模式的随机读性能有4-6倍的提升、随机写性能有有5-6倍的提升。提升效果接近商业软件FancyCache。
\end{enumerate}

\section{研究展望}
\label{sec:thesis_expectation}
随着云端计算的普及和海量数据时代的到来,计算机对现有存储系统的容量、数据可靠性和读写性能及提出了非常高的要求。固态硬盘作为一种高性能的存储设备由此产生,然而固态硬盘的价格昂贵,使得存储中心大多同时使用机械硬盘和固态硬盘进行混合存储。混合存储系统的研究也因此得到了研究人员的重视。从系统的测试结果和实现过程中的思考可以得出,论文提出的缓存系统在性能上还有很多可以提升的空间。具体可以归纳为以下几个点。


\subsubsection{结合固态硬盘特性的性能优化}

虽然固态硬盘的读写性能都高于机械硬盘的读写性能,但是固态硬盘的写性能相对于机械硬盘的优势并没有读操作的性能那么明显,而且当固态硬盘剩余的空闲空间不多时,还有可能出现严重影响写性能Write Cliff现象。针对固态硬盘写性能不是很高这一情况,下一步工作可以优化固态硬盘的写操作,整合多个连续的写请求,减少写操作的次数,提高缓存更新和缓存页面替换算法的运行效率。

\subsubsection{面向特殊场景的混合存储系统的研究}

从宏观层面看,应用程序发起的IO请求是随机且无法预测的。但是从每个程序的角度看,发起的IO请求却是有一定规律的,例如下载程序会从文件的多个位置开始顺序写文件。而且一般来说操作系统会在每个IO请求中标记发起进程的PID。缓存系统如果能够预测出某个进程的IO请求的发起模式,就可以根据模式调整缓存策略,提升系统的运行效率。因此,能够感知特殊IO请求的混合存储系统是一个新的有意义的方向。

\subsubsection{使用实时压缩算法提高缓存的空间利用率}

近几年来出现了许多优秀的用于工程实践的实时压缩算法,这类算法的压缩率虽然和主流的ZIP、RAR等软件存在差距,但是单线程的压缩速度却可以达到400MB/s以上,解压速度甚至能够达到1800MB/s。将实时压缩算法用于缓存系统,压缩率只要能够达到50\%,就可以节省一半的缓存空间。节约下来的缓存空间还可继续用于缓存数据,缓存系统命中率也会随之提高。Linux内核已使用zswap模块实时压缩交换到磁盘上的内存页面,相信这类压缩算法在缓存系统上的应用也将会得到普及。

%% ----------------------------------------------------------------------
%%% END OF FILE 
%% ----------------------------------------------------------------------
