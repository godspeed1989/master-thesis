%% ----------------------------------------------------------------------
%% START OF FILE
%% ----------------------------------------------------------------------
\chapter{总结与展望}
\label{cha:conclusions}

本论文首先介绍了固态硬盘和机械硬盘混合存储的技术背景和国内外存储设备厂商的相关研究工作,提出了使用固态硬盘作为缓存提升机械硬盘IO性能的解决方案。分析缓存技术的发展背景和支撑缓存系统的相关技术,分析不同类型存储介质的特点、混合存储技术的框架和缓存页面替换算法,进行论文方案可行的说明。

接下来,分析并详细介绍缓存系统实现中所用到的关键性技术,这些技术即包含学术和实践中已经成熟的技术,例如Windows系统的IO捕获、缓存映射策略,也包括论文提出的新技术解决方案,一种同时考虑访问时间和访问频率的缓存页面替换算法。

最后,给出缓存系统内部的实现细节和最终的测试结果。其中实现细节部分的介绍含缓存系统的整体设计架构、缓存算法通用接口和用户配置工具提供的参数。测试部分比较了本缓存系统和商业软件缓存系统对机械硬盘的性能提升效果。

\section{全文总结}
\label{sec:thesis_conclusion}

本文的主要研究成果和贡献可归纳为以下几点。
\begin{enumerate}


\end{enumerate}

\section{研究展望}
\label{sec:thesis_expectation}

\begin{enumerate}

\item 压缩

加入

\item 结合SSD的特性

\end{enumerate}
%% ----------------------------------------------------------------------
%%% END OF FILE 
%% ----------------------------------------------------------------------