%% ----------------------------------------------------------------------
%% START OF FILE
%% ----------------------------------------------------------------------

\begin{acknowledgments}

研究生毕设论文暂告收尾,这同时意味着我在西安电子科技大学研究生两年半的生活将要结束。回顾两年半来在西电的学习和生活,我深深感到自己不再是像本科期间那样只是机械的上课和学习新的理论知识,而是逐渐培养了自己独立进行科研和组织、管理技术项目的能力。可以将自己最宝贵的青春时光在西电绿树成荫、风景秀美的校园中,能在众多学富五车、才华横溢的老师们的熏陶下度过,实在是荣幸至极。

在进行毕业设计的实验和毕业论文的撰写期间,正是由于导师的谆谆教诲和同学们的出谋划策才使得我克服了一个又一个的难题,他们的鼓励和支持是我完成毕业设计和毕业论文的动力源泉。在此我还要特别感谢刘凯导师,他从论文的选题、文献的采集、系统框架的设计、实验的验证、文档结构的布局一直到论文的定稿,都无私的给予了我建议和指导,没有刘凯老师的栽培和教诲,我的毕业设计和毕业论文就不可能顺利完成。

感谢我的学长赵海星、李瑞,虽然只和他们相处了一年半的时间,但他们钻研科研、严谨治学的精神一直是我学习的楷模。还要感谢张劲同学提供的\LaTeX论文模板,使用他的论文模板使我能够专注于论文的写作而不必太过关心排版的细节。谢谢他们给予我的帮助和支持。最后,由衷地感谢我的父母十几年来为支持我的学业所付出的理解、支持和辛勤劳动。 

由于时间的仓促和本人的能力有限、经验不足,整篇文章肯定会存在尚未发现的缺点和错误。恳请阅读此篇论文的老师、同学多多予以指正。对百忙之中抽出宝贵时间对本论文进行评审的各位老师深表谢意。

\end{acknowledgments}

% ----------------------------------------------------------------------
%% END OF FILE
%% ---------------------------------------------------------------------