%% ----------------------------------------------------------------------
%% START OF FILE
%% ----------------------------------------------------------------------

\chapter{实验和结果分析}
\label{cha:exp_analysis}

\section{实验平台}
\label{sec:exp_platform}

操作系统 Windows Server 2008 R2

机械硬盘 250GB seagate st9250320as

固态硬盘 120GB crucial ct120m500ssd1

\section{实验方法}
\label{sec:exp_method}

\subsection{测试工具}
FIO

\subsection{工具参数}
\begin{lstlisting}
iodepth=8
\end{lstlisting}

\begin{lstlisting}
numjobs=4
\end{lstlisting}

\begin{lstlisting}
blocksize=[4k, 16k, 64k]
\end{lstlisting}

\begin{lstlisting}
rw=[randrw, randread, randwrite]
\end{lstlisting}

缓存卷大小 200MB

存储卷大小 2000MB

\section{实验结果}
\label{sec:exp_results}

\subsection{缓存命中率}

\subsection{写穿(Write Through)策略时的测试结果}
\begin{itemize}

\item 随机读速度测试

\begin{table}[H]
\centering
\caption{随机读速度(写穿法)}
\begin{tabular}{|c|c|c|c|}
\hline
\diagbox{大小(KB)}{速度(KB/s)} & HDD & SSD & HDD with SSD Cache \\ 
\hline 4 & 417 & 19264 & 2063 \\ 
\hline 16 & 1651 & 59735 & 6319 \\ 
\hline 64 & 5810 & 142304 & 13203 \\ 
\hline 
\end{tabular} 
\label{tab:wt-rand-read-test}
\end{table}

\item 随机写速度测试

\begin{table}[H]
\centering
\caption{随机写速度(写穿法)}
\begin{tabular}{|c|c|c|c|}
\hline
\diagbox{大小(KB)}{速度(KB/s)} & HDD & SSD & HDD with SSD Cache \\ 
\hline 4 & 1283 & 18620 & 1155 \\ 
\hline 16 & 5043 & 40634 & 4768 \\ 
\hline 64 & 16346 & 41615 & 15490 \\ 
\hline 
\end{tabular} 
\label{tab:wt-rand-write-test}
\end{table}

\item 随机读写:读速度测试

\begin{table}[H]
\centering
\caption{随机读写-读速度(写穿法)}
\begin{tabular}{|c|c|c|c|}
\hline
\diagbox{大小(KB)}{速度(KB/s)} & HDD & SSD & HDD with SSD Cache \\ 
\hline 4 & 420 & 14693 & 1590 \\ 
\hline 16 & 1657 & 46650 & 5327 \\ 
\hline 64 & 5598 & 105242 & 11929 \\ 
\hline 
\end{tabular} 
\label{tab:wt-randrw-read-test}
\end{table}

\item 随机读写:写速度测试

\begin{table}[H]
\centering
\caption{随机读写-写速度(写穿法)}
\begin{tabular}{|c|c|c|c|}
\hline
\diagbox{大小(KB)}{速度(KB/s)} & HDD & SSD & HDD with SSD Cache \\ 
\hline 4 & 45 & 1688 & 181 \\ 
\hline 16 & 180 & 5427 & 616 \\ 
\hline 64 & 599 & 11925 & 1371 \\ 
\hline 
\end{tabular} 
\label{tab:wt-randrw-write-test}
\end{table}

\end{itemize}

\subsection{写回(Write Back)策略时的测试结果}
\begin{itemize}

\item 随机读速度测试

\begin{table}[H]
\centering
\caption{随机读速度(写回法)}
\begin{tabular}{|c|c|c|c|}
\hline
\diagbox{大小(KB)}{速度(KB/s)} & HDD & SSD & HDD with SSD Cache \\ 
\hline 4 & 417 & 19264 & 2970 \\ 
\hline 16 & 1651 & 59735 & 9821 \\ 
\hline 64 & 5810 & 142304 & 32857 \\ 
\hline 
\end{tabular} 
\label{tab:wb-rand-read-tebst}
\end{table}

\item 随机写速度测试

\begin{table}[H]
\centering
\caption{随机写速度(写回法)}
\begin{tabular}{|c|c|c|c|}
\hline
\diagbox{大小(KB)}{速度(KB/s)} & HDD & SSD & HDD with SSD Cache \\ 
\hline 4 & 1283 & 18620 & 3459 \\ 
\hline 16 & 5043 & 40634 & 8697 \\ 
\hline 64 & 16346 & 41615 & 21794 \\ 
\hline 
\end{tabular} 
\label{tab:wb-rand-write-test}
\end{table}

\item 随机读写:读速度测试

\begin{table}[H]
\centering
\caption{随机读写-读速度(写回法)}
\begin{tabular}{|c|c|c|c|}
\hline
\diagbox{大小(KB)}{速度(KB/s)} & HDD & SSD & HDD with SSD Cache \\ 
\hline 4 & 420 & 14693 & 2725 \\ 
\hline 16 & 1657 & 46650 & 8693 \\ 
\hline 64 & 5598 & 105242 & 27696 \\ 
\hline 
\end{tabular} 
\label{tab:wb-randrw-read-test}
\end{table}

\item 随机读写:写速度测试

\begin{table}[H]
\centering
\caption{随机读写-写速度(写穿法)}
\begin{tabular}{|c|c|c|c|}
\hline
\diagbox{大小(KB)}{速度(KB/s)} & HDD & SSD & HDD with SSD Cache \\ 
\hline 4 & 45 & 1688 & 309 \\ 
\hline 16 & 180 & 5427 & 985 \\ 
\hline 64 & 599 & 11925 & 3221 \\ 
\hline 
\end{tabular} 
\label{tab:wb-randrw-write-test}
\end{table}

\end{itemize}

\section{结果与讨论}
\label{sec:results_and_comparation}

%% ----------------------------------------------------------------------
%%% END OF FILE
%% ----------------------------------------------------------------------